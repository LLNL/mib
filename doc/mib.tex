\documentclass{article}
\usepackage[pdftex]{graphicx}
\usepackage{epsfig}
\usepackage[hang]{subfigure}

\author{Andrew C. Uselton}
\title{
  MIB \\
  An MPI I/O Benchmark
}

\begin{document}

\maketitle

\begin{quote}
No testing has overtaken you that is not common to everyone. \\
1 Corinthians 10:13
\end{quote}


\begin{abstract} 
{\em Mib} is a simple MPI application that benchmarks I/O performance to a
parallel file system.  With it one may profile the timing of all the
write and read system calls performed during the test.  The level of
detail provided has been instrumental in identifying performance
bottlenecks in the Lustre file system.
\end{abstract}

\section{Introduction}\label{section.introduction}


{\em Mib} is a simple benchmark for testing file system performance,
especially of parallel file systems.  It resembles the {\em IOR}
benchmark\footnote{ftp://ftp.llnl.gov/pub/siop/ior/} in that it
carries out I/O from within a coordinated, parallel MPI application.
Following the idea of ``do only one thing, and do it well,'' {\em mib}
avoids the complexity and size associated with {\em IOR}.  It loads
quickly and is easily modified to suit specific needs.  Furthermore,
{\em mib} will produce a {\em profile} of the timing of every system
call.  These timings are synchronized and coordinated across all
tasks.  Post-processing tools\footnote{See the ``mib-tools'' RPM.} can
manipulate the profiles to give a graphical view of the dynamic
behavior of the file system under the load provided by {\em mib}.
These graphs have proved valuable in isolating performance
bottlenecks.

{\em Mib} is an outgrowth of Lustre\footnote{http://www.lustre.org/}
benchmarking efforts at LLNL\footnote{http://www.llnl.gov}, though its
operation does not require Lustre.  The environment that {\em mib}
uses is that of a Beowulf
cluster\footnote{http://en.wikipedia.org/wiki/Beowulf\_(computing)} with
MPI\footnote{http://www.mpi.org} 
coordinating tasks across the nodes of the cluster.  Each node has
access to a file syststem with a unified name space, i.e. they all see
the same files.  At LLNL the GNU/Linux-based clusters are additionally
controlled by the SLURM\footnote{http://www.llnl.gov/linux/slurm} resource
manager, and subsequent discussion will refer to SLURM tools
explicitly.  Other resource managers will operate in a similar
fashion.  

{\em Mib} sprang from perceived inadequacies of the {\em IOR}
benchmark, and it improves upon {\em IOR} in two ways.  First, its
simplicity allows for easy modification, a clear understanding of its
semantics, and a quick load time, compared to {\em IOR}.  Second, the
optional {\em system call profile} can reveal useful insights into
the dynamic behavior of a file system's I/O path without perturbing
that I/O.  On the other hand, {\em mib} is not a replacement for {\em
IOR}.  The latter has many useful modes of operation, including MPIIO
file system semantics\footnote{{\em Mib} confines itself to standard POSIX file system calls.}, shared file I/O, data
correctness checking, and scripted operation allowing many tests
within a single invocation.  {\em Mib} should be thought of as a
single purpose, precision tool that compliments the generality
available in {\em IOR}.

The initial development of {\em mib} was on the highly specialized
BlueGene/L supercomputer, and {\em mib} was instrumental in a
significant improvement in the bulk I/O performance of Lustre in that
environment.  Subsequent development has been on more conventional
i386 GNU/Linux Beowulf clusters, and in its current incarnation {\em
  mib} works on i386, ia64, x86\_64, powerpc, and ppc64 architectures
running Linux with Quadrics Elan3 and Elan4 interconnect 
hardware\footnote{http://www.qsw.com} and associated MPI libraries.
It is that version that is discussed here.  The BGL variant is under
continuing development, but is not generally available at this time.

The rest of this paper is organized as follows.
Section~\ref{section.normal} details the environment in which {\em
  mib} operates, gives common examples of its operation, and explains
what the results mean.  Section~\ref{section.usage} gives a
point-by-point explanation of each of {\em mib}'s options with examples.
Section~\ref{section.profiles} shows how to generate system call
profiles, explains what they mean, and gives examples in
which profiling revealed telling details about the operation of a
parallel file system.  Finally, Section~\ref{section.conclusion}
concludes with a summary of plans for its continued improvement.

\section{Mib Under Normal Operation}\label{section.normal}

Though {\em Mib} can run stand-alone on a computer without MPI, its main
use is to benchmark a parallel file system accessed from a Beowulf
cluster, in most cases running GNU/Linux.  The ALC cluster at LLNL is
typical and will be used for the examples that follow.  ALC has 960
dual-processor i386-based {\em compute nodes} identified as alc0 to
alc959.  Ranges of nodes are indicated with notation like: alc[0-959].
The nodes communicate via the QSW Elan3 high-speed interconnect.
Access to the nodes is controlled with the SLURM resource manager,
which organizes the compute nodes into resource pools called {\em
partitions}.  The {\em ltest} partition comprises alc[68-497].

Each node in the ltest partition mounts a Lustre file system with the
mount point \verb+/p/ga2+.  Access to that file system is provided by
a set of 16 {\em Portals routers}, alc[4-19], which communicate with
the compute nodes via the Elan interconnect and with the file system
servers via 32 GigE links (2 per router).  Each link attaches to a
Lustre OST (server) via a GigE switch, so the total available
bandwidth to the file system is about 4 GB/s.  

The file system service is monitored in real time with the {\em
  lmtd}\footnote{Unfortunately, that daemon is not available for
  circulation outside LLNL at the moment} daemon, a python-based data
  collector that feeds remote monitoring.  The {\em lwatch} utility
  gathers the server-side information and produces an aggregate file
  system data rate for writes and reads every five seconds.  {\em Mib}
  benchmark results can be compared against against {\em lwatch}
  results to get client-side versus server-side performance results (see Section~\ref{section.profiles}).   

\begin{figure}
  \begin{verbatim}
    srun -N8 -p ltest mib -t /p/ga2/lustre-test/mib/newfiles
    |    |      |     |       |
    |    |      |     |       ---target directory
    |    |      |     -----------MPI application
    |    |      -----------------partition (group of target nodes)
    |    ------------------------number of tasks
    -----------------------------SLURM MPI launcher
  \end{verbatim}
\caption{A typical {\em mib} command line}
\label{figure.typical}
\end{figure}

In the LLNL Linux clusters environment the SLURM command {\em srun}
launches MPI applications.  Figure~\ref{figure.typical} gives a
typical invocation, which would run 8 MPI tasks on nodes taken from
the ltest partition.  The {\em mib} executable behaves according to
the psuedocode depicted in Figure~\ref{figure.pseudo}.

\begin{figure}
\begin{tabbing}
parse command line \\
initialize data structures, MPI, timers \\
write \= test \\
  \> barrier \\
  \> $t_{(i, \mbox{open})}$ \\
  \> open file$_i$\\
  \> barrier \\
  \> $t_{(i, 0)}$ \\
  \> while \= not done \\
  \>  \>  $w_{(i, j)} = $ write \\
  \>  \>  $t_{(i, j)}$ \\
  \>  \>  $j++$ \\
  \> fsync \\
  \> barrier \\
  \> $t_{(i, \mbox{fsync})}$ \\
  \> gather $W = \sum_{(i, j)} w_{(i, j)}$ \\
  \> gather $T_W = \max_i t_{(i, \mbox{fsync})} - \min_i t_{(i, 0)}$ \\ 
  \> close \\
  \> $t_{(i, \mbox{close})}$ \\
read test \\
  \> barrier \\
  \> $t_{(i, \mbox{open})}$ \\
  \> open file$_i$\\
  \> barrier \\
  \> $t_{(i, 0)}$ \\
  \> while \= not done \\
  \>  \>  $r_{(i, j)} = $ read \\
  \>  \>  $t_{(i, j)}$ \\
  \>  \>  $j++$ \\
  \> last $= j$ \\
  \> barrier \\
  \> gather $R = \sum_{(i, j)} r_{(i, j)}$ \\
  \> gather $T_R = \max_i t_{(i, \mbox{last})} - \min_i t_{(i, 0)}$ \\ 
  \> close \\
  \> $t_{(i, \mbox{close})}$ \\
report (from $task_0$ only) \\
  \> ``Write rate is $W/T_W$'' \\
  \> ``Read rate is $R/T_R$'' \\
\end{tabbing}
\caption{Pseudo-code for {\em mib}: $task_i (i \in \{1 \ldots N\})$}
\label{figure.pseudo}
\end{figure}


The I/O continues for the default amount of time or the default number
of transfers (whichever limit is reached first), and would involve
system calls using the default amount of data.  Optional activities
detailed in Section~\ref{section.usage} add a little to the above, but
the code is simple enough that its semantics is clear and it can be
easily extended, for instance if metadata activity is of interest.

The most common use of {\em mib} would include a limit on the number
of system calls, a limit on time, and a system call size to override
the default values.  For instance, to perform, per task, no more than
4500 system calls of $4MB$ bytes each and to cease issuing system
calls after 300 seconds, the command and its result would look like
this:
{\small
  \begin{verbatim}
    srun -N8 -p ltest mib -t /p/ga2/lustre-test/mib/newfiles \
    -l 4500 -L 300 -s 4m
    Fri Mar 17 22:50:59 2006      8    4M  4500   300	897.07     566.20
  \end{verbatim}
}
\pagebreak
The ``-H'' option provides an informative header:
{\small
  \begin{verbatim}
    srun -N8 -p ltest mib -t /p/ga2/lustre-test/mib/newfiles \
    -l 4500 -L 300 -s 4m -H
    
    date            tasks  xfer  call  time      write       read
    limit limit       MB/s       MB/s
    ------------------------ ------ ----- ----- ----- ---------- ----------
    Wed Apr 12 15:17:43 2006      8    4M  4500   300    1096.89     665.96
  \end{verbatim}
}

The MPI interface provides a unique {\em rank} to each task, which is
a number from $0$ to $7$ in this case.  The eight tasks in
the example each opens its own file with the rank giving the files
unique names:
{\small
  \begin{verbatim}
    ls -l /p/ga2/lustre-test/mib/newfiles
    total 147600256
    -rw-r--r--  1 auselton auselton 18874368000 Apr 12 15:14 mibData.00000000
    -rw-r--r--  1 auselton auselton 18874368000 Apr 12 15:13 mibData.00000001
    -rw-r--r--  1 auselton auselton 18874368000 Apr 12 15:14 mibData.00000002
    -rw-r--r--  1 auselton auselton 18874368000 Apr 12 15:14 mibData.00000003
    -rw-r--r--  1 auselton auselton 18874368000 Apr 12 15:14 mibData.00000004
    -rw-r--r--  1 auselton auselton 18874368000 Apr 12 15:13 mibData.00000005
    -rw-r--r--  1 auselton auselton 18874368000 Apr 12 15:13 mibData.00000006
    -rw-r--r--  1 auselton auselton 18874368000 Apr 12 15:13 mibData.00000007
  \end{verbatim}
} 
Note that the results in these two runs differ.  The Lustre
enviornment was changed between the two.  In the first test, the
target directory was configured to stripe each file over a single
server.  In that case 8 servers will be active, each with a single
file corresponding to a single task.  The maximum bandwidth for the
eight servers together is $1GB/s$.  The measured write rate is $897
MB/s$, or about $88\%$ of ``wire speed.''  The read rate is a little
lower at $566 MB/s$.

The read target for node $i \in \{0..\mbox{$size$}\}$ is calculated as
$(i + \mbox{$size$}/2) \% \mbox{$size$}$.  As long as there is more
than one node in the MPI job\footnote{If there are multiple tasks per
  node and ranks are assigned ``round-robin'' the write and read
  targets might still be on the same node.  This has not been observed
in practice but should be kept in mind for other MPI environments.
The choice of read target is easily modified in the code if such a
change is needed.} the file that a task reads will have
been produced by a different node, thus defeating any buffer cache
effects at the compute nodes.  

In the second test the files used the default of 2 stripes per file,
so 16 servers would be involved\footnote{If other files were being
created at the same time the allocations might not be smoothly
distributed over the servers, in which case there would be fewer
servers participating.  This would both reduce the available bandwidth
and unevenly burden one or a few servers making the results skewed
toward the unlucky tasks.}.  Note that the data rates were a bit
higher in the second test (with the header), though not double.

The files all ended up the same size: $17.6GB = 4500 * 4MB$.  That
indicates that every task was able to complete all 4500 transfers in
the 300 seconds alloted.  If some tasks finish well before others then
the measurement of the data rate for the whole job can be skewed by
the poor performance of the slowest nodes or servers.  An alternate
measurement would be to make the number of transfers larger than what
even the fastest task can accomplish in the given period.  Then the
measured data rate represents the full bandwidth available to the job.
This is called ``stonewalling''.  
{\small
  \begin{verbatim}
    srun -N8 -p ltest mib -t /p/ga2/lustre-test/mib/newfiles \
    -l 4500 -L 60 -s 4m -H
               date            tasks  xfer  call  time      write       read
                                            limit limit       MB/s       MB/s
    ------------------------ ------ ----- ----- ----- ---------- ----------
    Wed Apr 12 15:31:40 2006      8    4M  4500    60    1137.30     842.72
  \end{verbatim}
}
After such a stonewalling test the file sizes are all different
from each other. 
{\small
  \begin{verbatim}
ls -l /p/ga2/lustre-test/mib/newfiles
total 70069200
-rw-rw-r--  1 auselton auselton 8409579520 Apr 12 15:30 mibData.00000000
-rw-rw-r--  1 auselton auselton 9089056768 Apr 12 15:30 mibData.00000001
-rw-rw-r--  1 auselton auselton 9072279552 Apr 12 15:30 mibData.00000002
-rw-rw-r--  1 auselton auselton 8711569408 Apr 12 15:30 mibData.00000003
-rw-rw-r--  1 auselton auselton 9244246016 Apr 12 15:30 mibData.00000004
-rw-rw-r--  1 auselton auselton 9281994752 Apr 12 15:30 mibData.00000005
-rw-rw-r--  1 auselton auselton 8782872576 Apr 12 15:30 mibData.00000006
-rw-rw-r--  1 auselton auselton 9089056768 Apr 12 15:30 mibData.00000007
  \end{verbatim}
}
If files (or file segments) are unevenly distributed across servers
some tasks can experience more contention than others leading to I/O
rates two or more times slower in some tasks than in others.  Clearly, that
did not happen in this case.

For parallel benchmarking the foregoing is enough to get results from
{\em mib}.  The next section details all the options available.  

\section{Usage}\label{section.usage} 

\begin{figure}
{\small
  \begin{verbatim}
usage: mib [b::EFIhHl:L:Mnp:PrRs:St:VW]
See the man page for "long_opts" equivalents.
    -b [<gran>]     :  Random seeks (optional granularity) before each read.
    -E              :  Show environment of test in output.
    -F              :  Mib does not normally allow I/O to any FS
                    :  but Lustre.  This overrides the "safety".
    -I              :  Show intermediate values in output.
    -h              :  Print this message and exit.
    -H              :  Show headers in output.
    -l <call_limit> :  Issue no more than this many system calls.
    -L <time_limit> :  Do not issue new system calls after this many
                    :    seconds (limits are per phase for write and
                    :    read phases).
    -M              :  Do not use MPI even if it is available.
    -n              :  Create new files if files were already present
                    :    (will always create new files if none were
                    :    present).
    -p <profiles>   :  Output system call timing profiles to 
                    :    <profiles>.write and <profiles>.read.
    -P              :  Show progress bars during testing.
    -r              :  Remove files when done.
    -R              :  Only perform the read test.
    -s <call_size>  :  Use system calls of this size (default 512k).
                    :    Numbers may use abbreviations k, K, m, or M.
    -S              :  Show signon message, including date, program
                    :    name, and version.
    -t <test_dir>   :  Required. I/O transactions to and from this
                    :    directory.
    -V              :  Print the version and exit.
    -W              :  Only perform the write test
                    :    (if -R and -W are both present both tests 
                    :     will run, but that's the default anyway).
  \end{verbatim}
}
\caption{The {\em usage} of {\em mib}}
\label{figure.usage}
\end{figure}

The simplest {\em mib} command line, {\em mib -h}, is useful on a node
where MPI is not even installed.  It gives the ``usage'' as depicted
in Figure~\ref{figure.usage}.  Similarly, {\em mib -V} gives the
version including the architecture on which it was compiled.  In the
case of the tests presented in this document the version is {\em
mib-1.9.2-i386}.  One may run {\em mib} on a stand-alone node with or
without MPI, but there is no compelling reason to do so.  There are
other non-parallel I/O benchmarks that do an excellent
job\footnote{http://www.iozone.org/}, so the rest of this discussion
will continue to use the ALC cluster introduced in
Section~\ref{section.normal}.

\pagebreak
The next example adds a report on the environment of the test and an
initial ``signon'' message with the date and version number.
{\small
  \begin{verbatim}
srun -N8 -p ltest mib -t /p/ga2/lustre-test/mib/newfiles \
          -l 4500 -L 60 -s 4m -SHE
Testing a file system of type "S_MAGIC_LUSTRE" (f_type = 0xbd00bd0)
Using MPI


mib-1.9  Wed Apr 12 15:48:46 2006

cluster                  = alc
new                      = false
remove                   = false
nodes                    = 8
testdir                  = /p/ga2/lustre-test/mib/newfiles
call_limit               = 4500
call_size                = 4194304
time_limit               = 60
tasks                    = 8
write_only               = false
read_only                = false
profiles                 = no
use_node_aves            = false
show_signon              = true
show_headers             = true
show_environment         = true
show_progress            = false
show_intermediate_values = false
          date            tasks  xfer  call  time      write       read
                                      limit limit       MB/s       MB/s
------------------------ ------ ----- ----- ----- ---------- ----------
Wed Apr 12 15:50:46 2006      8    4M  4500    60    1183.29     840.95
  \end{verbatim}
}
The ``-t \verb+<target>+'' is a required parameter.  Leaving it off
has consequences in an environment where home directories are NFS
mounted.  A cluster of computers all doing I/O to the current
working directory could cause a ``denial of service attack'' on the
NFS server.
\begin{verbatim}
mib
Mib requires a "-t <test_dir>" argument.
\end{verbatim}
Also see below about testing NFS file systems.  

\pagebreak
With the ``-I'' and ``-P'' options you get intermediate values as they
are calculated and a progress meter visually tracking the percent
complete of the test:
{\small
  \begin{verbatim}
srun -N8 -p ltest mib -t /p/ga2/lustre-test/mib/newfiles \
          -l 4500 -L 60 -s 4m -HIP

write phase
should last about this long---------------------->
************************************************
Call 2276 was the last write recorded

After 2276 calls and 60.090934 seconds
71144.000000 MB written in 60.080710 seconds
Aggregate write rate =    1184.14

read phase
should last about this long---------------------->
************************************************
Call 1602 was the last read recorded

After 1602 calls and 60.038198 seconds
50476.000000 MB read in 60.034850 seconds
Aggregate read rate =     840.78
          date            tasks  xfer  call  time      write       read
                                      limit limit       MB/s       MB/s
------------------------ ------ ----- ----- ----- ---------- ----------
Wed Apr 12 15:55:43 2006      8    4M  4500    60    1184.14     840.78
  \end{verbatim}
}
The progress meter is a rudimentary convenience and not intended to be
a precision measure.  It can be fooled by caching and by extremely
slow or uneven system call completion.

One may optionally run just the write test or just the read test:
{\small
  \begin{verbatim}
srun -N8 -p ltest mib -t /p/ga2/lustre-test/mib/newfiles \
          -l 4500 -L 60 -s 4m -W  
Wed Apr 12 15:58:25 2006      8    4M  4500    60    1194.42       0.00

srun -N8 -p ltest mib -t /p/ga2/lustre-test/mib/newfiles \
          -l 4500 -L 60 -s 4m -R
Wed Apr 12 15:59:51 2006      8    4M  4500    60       0.00     874.30
  \end{verbatim}
}
During a ``-R'' test if there are no files at the target location {\em
mib} will fail with an explanatory message.  Putting both ``-W'' and
``-R'' on the command line produces both tests, which is the default
anyway.

\pagebreak
There is an optional random seek available in the read test.  The
``granularity'' can be set with the option's parameter:
{\small
  \begin{verbatim}
srun -N8 -p ltest mib -t /p/ga2/lustre-test/mib/newfiles \
          -l 4500 -L 60 -s 4m -R -b
Wed Apr 12 16:02:52 2006      8    4M  4500    60       0.00     404.36

srun -N8 -p ltest mib -t /p/ga2/lustre-test/mib/newfiles \
          -l 4500 -L 60 -s 4m -R -b4k
Wed Apr 12 16:04:08 2006      8    4M  4500    60       0.00    4166.64
  \end{verbatim}
}
By default, the seek can be to any byte offset, if the option includes
a granularity then the seek is to some random multiple of that
value.  It is also worth noting that the latter test read from cache,
and was quite fast because of it.

The ``-r'' option removes the files at the end of the test.  {\em Mib}
will always create files if they are not already present, but by
default will use files that do already exist.  To force the creation
of new files use ``-n''.  In the next sequence of tests files are
removed and recreated:
{\small
  \begin{verbatim}
srun -N8 -p ltest mib -t /p/ga2/lustre-test/mib/newfiles \
          -l 4500 -L 60 -s 4m -r     
Wed Apr 12 16:10:06 2006      8    4M  4500    60    1185.60 832.80

ls -l /p/ga2/lustre-test/mib/newfiles
total 0

 srun -N8 -p ltest mib -t /p/ga2/lustre-test/mib/newfiles \
          -l 4500 -L 60 -s 4m
Wed Apr 12 16:14:00 2006      8    4M  4500    60    1177.24     840.95

ls -l /p/ga2/lustre-test/mib/newfiles
total 72521016
-rw-rw-r--  1 auselton auselton 9147777024 Apr 12 16:13 mibData.00000000
-rw-rw-r--  1 auselton auselton 9537847296 Apr 12 16:13 mibData.00000001
-rw-rw-r--  1 auselton auselton 9051308032 Apr 12 16:13 mibData.00000002
-rw-rw-r--  1 auselton auselton 9676259328 Apr 12 16:13 mibData.00000003
-rw-rw-r--  1 auselton auselton 9340715008 Apr 12 16:13 mibData.00000004
-rw-rw-r--  1 auselton auselton 9156165632 Apr 12 16:13 mibData.00000005
-rw-rw-r--  1 auselton auselton 9550430208 Apr 12 16:13 mibData.00000006
-rw-rw-r--  1 auselton auselton 8728346624 Apr 12 16:13 mibData.00000007
  \end{verbatim}
}
The files have been created anew.

\pagebreak
Continuing the example of the previous page a run with ``-n'' recreates
the files.  Note the changed timestamps.
{\small
  \begin{verbatim}
 srun -N8 -p ltest mib -t /p/ga2/lustre-test/mib/newfiles \
          -l 4500 -L 60 -s 4m -n
Wed Apr 12 16:17:29 2006      8    4M  4500    60    1159.59     842.01

ls -l /p/ga2/lustre-test/mib/newfiles
total 71450912
-rw-rw-r--  1 auselton auselton 8757706752 Apr 12 16:16 mibData.00000000
-rw-rw-r--  1 auselton auselton 8753512448 Apr 12 16:16 mibData.00000001
-rw-rw-r--  1 auselton auselton 9261023232 Apr 12 16:16 mibData.00000002
-rw-rw-r--  1 auselton auselton 8854175744 Apr 12 16:16 mibData.00000003
-rw-rw-r--  1 auselton auselton 9063890944 Apr 12 16:16 mibData.00000004
-rw-rw-r--  1 auselton auselton 9441378304 Apr 12 16:16 mibData.00000005
-rw-rw-r--  1 auselton auselton 9722396672 Apr 12 16:16 mibData.00000006
-rw-rw-r--  1 auselton auselton 9240051712 Apr 12 16:16 mibData.00000007
  \end{verbatim}
}

When MPI is not available {\em mib} will try to carry out a test
anyway.  To test this behavior one may deliberately suppress MPI with
the ``-M'' option.  This behavior is also available if SLURM is not
present, not working, or one wants to suppress SLURM for some reason.
The {\em pdsh}\footnote{http://www.llnl.gov/linux/pdsh/} command is a
parallel remote shell utility that can be used to launch {\em mib},
again without MPI.  The next example shows a test running under SLURM
but without MPI.  Each task reports results, since it cannot
coordinate with the other tasks:
{\small
  \begin{verbatim}
srun -N8 -p ltest mib -t /p/ga2/lustre-test/mib/newfiles \
          -l 4500 -L 60 -s 4m -M
     4 Wed Apr 12 16:28:16 2006      8    4M  4500    60     147.53     106.40
     5 Wed Apr 12 16:28:16 2006      8    4M  4500    60     150.00     106.73
     6 Wed Apr 12 16:28:16 2006      8    4M  4500    60     153.47     106.00
     3 Wed Apr 12 16:28:16 2006      8    4M  4500    60     147.13     106.40
     7 Wed Apr 12 16:28:16 2006      8    4M  4500    60     144.40     106.20
     2 Wed Apr 12 16:28:16 2006      8    4M  4500    60     151.53     105.80
     1 Wed Apr 12 16:28:16 2006      8    4M  4500    60     150.27     106.07
     0 Wed Apr 12 16:28:16 2006      8    4M  4500    60     141.67      98.00
  \end{verbatim}
}

\pagebreak
Next we run a test without SLURM under {\em pdsh}.  Note that every
task thinks it is the only one and they all produce reports:
{\small
  \begin{verbatim}
 pdsh -w alc[68-75] mib -t /p/ga2/lustre-test/mib/newfiles \
          -l 4500 -L 60 -s 4m -M -SHE
alc68: Testing a file system of type "S_MAGIC_LUSTRE" (f_type = 0xbd00bd0)
alc68: Command line forbade the use of MPI
alc68: 
alc68: 
alc68: mib-1.9  Wed Apr 12 16:31:37 2006
alc68: 
alc68: cluster                  = 
alc68: new                      = false
alc68: remove                   = false
alc68: nodes                    = 1
alc68: testdir                  = /p/ga2/lustre-test/mib/newfiles
alc68: call_limit               = 4500
alc68: call_size                = 4194304
alc68: time_limit               = 60
alc68: tasks                    = 1
alc68: write_only               = false
alc68: read_only                = false
alc68: profiles                 = no
alc68: use_node_aves            = false
alc68: show_signon              = true
alc68: show_headers             = true
alc68: show_environment         = true
alc68: show_progress            = false
alc68: show_intermediate_values = false
[ other nodes report similarly]
alc68:   task           date            tasks  xfer  call  time      write       read
alc68:                                              limit limit       MB/s       MB/s
alc68: ------ ------------------------ ------ ----- ----- ----- ---------- ----------
alc68:     68 Wed Apr 12 16:33:37 2006      1    4M  4500    60     139.00      99.40
[ and so on]
  \end{verbatim}
}

\pagebreak
Also note the file names resulting from the {\em pdsh} run.
{\small
  \begin{verbatim}
ls -l /p/ga2/lustre-test/mib/newfiles
total 145390528
-rw-rw-r--  1 root root 8912896000 Apr 12 16:27 mibData.00000000
-rw-rw-r--  1 root root 9453961216 Apr 12 16:27 mibData.00000001
-rw-rw-r--  1 root root 9533652992 Apr 12 16:27 mibData.00000002
-rw-rw-r--  1 root root 9256828928 Apr 12 16:27 mibData.00000003
-rw-rw-r--  1 root root 9281994752 Apr 12 16:27 mibData.00000004
-rw-rw-r--  1 root root 9441378304 Apr 12 16:27 mibData.00000005
-rw-rw-r--  1 root root 9722396672 Apr 12 16:27 mibData.00000006
-rw-rw-r--  1 root root 9240051712 Apr 12 16:27 mibData.00000007
-rw-r--r--  1 root root 8745123840 Apr 12 16:32 mibData.00000068
-rw-r--r--  1 root root 9302966272 Apr 12 16:32 mibData.00000069
-rw-r--r--  1 root root 9076473856 Apr 12 16:32 mibData.00000070
-rw-r--r--  1 root root 9089056768 Apr 12 16:32 mibData.00000071
-rw-r--r--  1 root root 9458155520 Apr 12 16:32 mibData.00000072
-rw-r--r--  1 root root 9466544128 Apr 12 16:32 mibData.00000073
-rw-r--r--  1 root root 9286189056 Apr 12 16:32 mibData.00000074
-rw-r--r--  1 root root 9466544128 Apr 12 16:32 mibData.00000075
  \end{verbatim}
}

Since MPI can't coordinate the activity, each task reports as though
it were the only one.  Rather than having each task decide it is
``task 0'', in this mode a task gets its number from the hostname if
it can.  The latter test above would figure out it doesn't have MPI
even without the ``-M'', since SLURM will not have provided the needed
``capability'' (that being the Elan term for needed infrastructure for
MPI operation). 

As mentioned above, {\em mib} will not willingly run an I/O test to an NFS
mounted home directory.  This is a safety feature.  Other file systems
are similarly protected.  If such a test is really needed the ``-F''
option will force the test even though it is a bad idea.  An example
is not provided because it tends to be quite hard on the home
directories file system.  

Finally, the ``-p \verb+<profile>+'' option will create two new files,
``\verb+<profile>+.write'' and ``\verb+<profile>+.read'' with the timing of each
system call in each task.  
\begin{verbatim}
srun -N8 -p ltest mib -t /p/ga2/lustre-test/mib/newfiles \
          -l 4500 -L 60 -s 4m -p /g/g0/auselton/alc/testing/profile
Wed Apr 12 16:46:13 2006      8    4M  4500    60    1242.37 822.55

ls -l /g/g0/auselton/alc/testing/       
total 324
-rw-r--r--  1 root root 130799 Apr 12 16:46 profile.read
-rw-r--r--  1 root root 189824 Apr 12 16:45 profile.write
\end{verbatim}
Each of the two files contains a table of timestamp values.  All the
time values are relative to a ``zero'' coordinated (via an MPI
barrier) at the beginning of the test and adjusted for any skew in the
``MPI\_Wtime()'' global clock readings.  Each column holds timing
values for a task.  Each row gives the time at which the given task
completed the given system call (i.e. the MPI\_Wtime() at which it
returned).  Since some tasks don't make as much progress as others the
table is zero-filled to have the same number of entries in every
column.

Despite the long list of command line options and their examples
presented here, the {\em mib} program is a simple and easy to use
tool.  The results it provides are easy to interpret.  The source code
is easy to modify to include any special purpose analysis deemed
necessary.  The next section discusses how to use the profiles and
what they mean.

\section{Generating system call profiles}\label{section.profiles}

The other common use for {\em mib}, beside generating write and read
data transfer rates, is to generate a system call timing profile.  As
mentioned at the end of Section~\ref{section.usage}, invoking the
``-p \verb+<profile>+'' option creates two tables of timing values,
one for writes and one for reads.  The perl script ``composite.pl''
(see the {\em mib-tools} RPM) uses the two profiles to create a graph
depicting the dynamic behavior of the benchmark.
Figure~\ref{composite.fig} is an example of such a graph taken from
the {\em ALC} cluster during a run at scale (800 tasks on 400 nodes).

\begin{figure}
  \includegraphics[scale=0.50]{sample.png}
\caption{A graph of the system call profiles and other information}
\label{composite.fig}
\end{figure}

The graph shows 300 seconds of writes at $1743 MB/s$ and another 300
seconds of reads at about $326 MB/s$.  The composite graph has more
than just the system call timings, which are the high frequency lines
in the graph.  The ``square'' lines mark the data rate and duration of the
writes and reads, as reported by {\em mib}.  The lower frequency varying
lines that closely track the system call timings are independent
readings taken every five seconds from the servers.  The later
readings come from {\em lwatch}, a separate ({\em Python}-based)
script that samples 
the server monitoring daemons mentioned in Section~\ref{section.normal}.

System call profile graphs allow one to visualize the dynamic behavior
of the file system in a way that benchmark results or tables of
numbers cannot.  Graphs like the one depicted in
Figure~\ref{composite.fig} have been instrumental in identifying,
fixing, or explaining the following effects: 
\begin{description}
\item [Stragglers] If one or a few tasks take much longer to complete
  their I/O it is readily apparent from the graph (don't do
  stonewalling, of course).
\item [Discarded Read-ahead]  If the server-side rates are much
  higher than the client-side rates during reads the difference is
  probably read-ahead data that has been discarded unused.
\item [Read Cache Effects] Conversely, if the clients see read rates
  much higher than the servers that data was probably already in the
  client cache.
\item [Write Cache Effect] A spike in client write rates at the
  beginning of a test (Figure~\ref{composite.fig} shows this) is
  likely the client buffer cache filling up.  
\item [Periodic Server Metadata Activity] Long I/O can show periodic
  dips in performance that reveal underlying server metadata activity.
\item [Dead Disks] If a disk is going bad and the RAID controller is
  slow to notice and fail the disk this will show up as a period of
  reduced I/O that stands out in the graph but would only show up as
  ``poor'' performance in a benchmark.
\item [Read, Modify, Write] Small writes instigate reads in the Linux
  VFS as pages need to be read in before they can be modified.  This
  also shows up clearly in the graph whereas it would only appear in a
  benchmark result as ``poor'' performance.
\end{description}

There is a wealth of information to be gained by examining the dynamic
behavior of the file system during a test.  All of the above examples
were noticed first as phenomena in the graphs and later explained or
fixed in the file system code.  


\section{Conclusion}\label{section.conclusion}

{\em Mib} is an excellent precision tool for understanding the dynamic
performance of a parallel file system.  There is a great deal more
that could be done to improve the use and understanding of {\em mib}
results.  It would be helpful to develop a ``bestiary'' of graphs
depicting various bad or exceptional behaviors along with explanations
for each.  It would be helpful to have a comprehensive theory of
parallel I/O against which to compare real world results.  The server
monitoring tools are a valuable compliment to {\em mib} analysis, and
it would be nice to be able to distribute them.  Finally, there are
other dimensions to the parallel I/O problem besides bulk write and
read data rates.  Metadata performance would be interesting to try to
measure and visualize in a similar fashion.

\end{document}
